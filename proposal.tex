\documentclass[a4paper, 11pt]{article}

\usepackage{kotex} % Comment this out if you are not using Hangul
\usepackage{fullpage}
\usepackage{hyperref}
\usepackage{amsthm}
\usepackage[numbers,sort&compress]{natbib}

\theoremstyle{definition}
\newtheorem{exercise}{Exercise}

\begin{document}
%%% Header starts
\noindent{\large\textbf{IS-521 Activity Proposal}\hfill
                \textbf{안형철}} \\
         {\phantom{} \hfill \textbf{Hyeongcheol-An}} \\
         {\phantom{} \hfill Due Date: April 15, 2017} \\
%%% Header ends

\section{Activity Overview}

RSA 알고리즘은 Side-channel attack의 하나인 timing-attack에 취약하다. 
현재 timing-attack을 막을 수 있는 기법이 나와 있지만, 
실제 안전하지 않은 RSA알고리즘을 대상으로 timing-attack을 시도하고 
이를 막는 대응책을 구현하고 테스트하는 것이 본 activity의 목적이다.

\section{Exercises}

본 activity는 총 3개의 exercise로 구성되어 있다.

\begin{exercise}

  Timing attack이 가능한 RSA 소스 코드 \cite{RSAsourcecode} 를 가지고 side-channel attack을 시도한다. 
  이 exercise에서 예상되는 결과는 secret key를 찾는 것이다.

\end{exercise}

\begin{exercise}

  앞선 exercise에서의 RSA 소스코드를 수정하는 것이 주 목적이다.
  현재 나와 있는 timing attack에 대한 대응책 또는 개선된 방법으로 RSA 소스 코드를 수정한다. 
  기본적인 아이디어는 constant time or random time encryption이다.

\end{exercise}

\begin{exercise}

  Exercise 2에서 수정한 RSA 알고리즘을 exercise 1에서 시도했던 방법으로 timing attack을 시도할 경우
  secret key를 찾는 것이 불가능하다는 것을 확인한다.

\end{exercise}

\section{Expected Solutions}

Exercise 1을 성공했을 경우 secret key를 알아낼 수 있다. 
Exercise 2에서 constant time operation 또는 random time operation을 적용하여 side-channel attack을 막을 수 있다. 
Exercise 3의 테스트로 실제 개선한 알고리즘이 정확히 동작하는지 확인할 수 있다.



\bibliography{references}
\bibliographystyle{plainnat}

\end{document}
